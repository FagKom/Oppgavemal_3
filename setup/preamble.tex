% =============================================================================
% Language, punctuation and fonts
% =============================================================================
\usepackage[utf8]{inputenc} % inputs
\usepackage[english]{babel} % english language, e.e. \chapter gives "chapter 1" 
\usepackage{csquotes}       % use \textquotes{} for proper "" in text

% =============================================================================
% Page color
% =============================================================================
% use this to generate grey pages to work in poorly lit rooms
%\usepackage{xcolor}
%    \pagecolor[rgb]{0.5, 0.5, 0.5}
%    \color[rgb]{1, 1, 1}

% =============================================================================
% Math
% =============================================================================
% Math needs to be loaded before amsthm, so QED can hook into align*
\usepackage{amssymb, mathrsfs, mathtools}   % mathematical symbols, e.g. \varnothing
\usepackage{amsthm}                         % more stuff
\usepackage{tikz}                           % drawings
\usepackage{tikz-cd}                        % easy commutative diagrams

% =============================================================================
% Misc
% =============================================================================
\usepackage{lipsum}
\usepackage{fancyhdr}   % creates fancy headers and footers
\usepackage{appendix}
\usepackage{enumitem}
\usepackage{subcaption} 
\usepackage{todonotes}
\usepackage{emptypage}  % removes page numbering on empty pages


% =============================================================================
% Bibliography
% =============================================================================

\usepackage[
    style=numeric, 
    sorting=nyt, 
    giveninits=true, 
    eprint=false, 
    url=false, 
    doi=false,
    isbn=false,
    natbib=true
    ]{biblatex}

    \addbibresource{references.bib}

% =============================================================================    
% Hyper references
% =============================================================================

\usepackage[
    bookmarks=true % adds a navigation menu usually shown in a left panel of the pdf-reader
    ]{hyperref}
    
    \hypersetup{
        colorlinks = true,
        %citecolor=green,
        %filecolor=black,
        %linkcolor=black,
        %urlcolor=purple
    }
    
% =============================================================================    
% References % must be after hyperref
% =============================================================================
%Converts the hyperlink "(3.4)" into "eq. (3.4)". 
\usepackage[
    %nameinlink, % makes the whole reference clickable
    %noabbrev    % makes " eq. (3.4)" into "equation (3.4)")
    ]{cleveref}

    % The format is:
    %\crefname{type}{singular}{plural}
    %\Crefname{type}{singular}{plural} % this is for capitalized references
    
    \crefname{mydef}{definition}{definitions}
    \Crefname{mydef}{Definition}{Definitions}
    
    \crefname{ex}{example}{examples}
    \Crefname{ex}{examples}{Examples}
    
    \crefname{prop}{proposition}{propositions}
    \Crefname{prop}{Proposition}{Propositions}
    
    \crefname{lemma}{lemma}{lemmas}
    \Crefname{lemma}{Lemma}{Lemmas}
    
    \crefname{thm}{theorem}{theorems}
    \Crefname{thm}{Theorem}{Theorems}
    
    \crefname{cor}{corollary}{corollaries}
    \Crefname{cor}{Corollary}{Corollaries}
    
    \crefname{remark}{remark}{remarks}
    \Crefname{remark}{Remark}{Remarks}
        
    \crefname{myquestion}{question}{questions}
    \Crefname{myquestion}{Question}{Questions}
    
% =============================================================================
% Theorems, definitions, lemma and stuff % must be after cleverref
% =============================================================================
% Note that we use the same counter for definitions, theorems, lemmas, propositions and corollaries
\theoremstyle{plain}
    \newtheorem{thm}{Theorem}[chapter]
    \newtheorem*{thm*}{Theorem}
    \newtheorem{lemma}[thm]{Lemma}
    \newtheorem{cor}[thm]{Corollary}
    \newtheorem{prop}[thm]{Proposition}
 
\theoremstyle{definition}
    \newtheorem{ex}[thm]{Example}
    \newtheorem{mydef}[thm]{Definition}
    \newtheorem{remark}[thm]{Remark}
    \newtheorem*{remark*}{Remark}
    \newtheorem{constr}[thm]{Construction}

% =============================================================================
% Plain style indents
% =============================================================================
% Fixes indents for the plain style, makes it cursive. E.g.: theorems
\makeatletter
\def\th@plain{
   \itshape
    \thm@headfont{\scshape}
    \let\thm@indent\noindent
}
\makeatother

% =============================================================================
% Definition style indents
% =============================================================================
% Fixes indents for the definition style, makes only the header cursive
\makeatletter
\def\th@definition{
    \thm@headfont{\scshape}
    \let\thm@indent\noindent
}
\makeatother

% =============================================================================
% Proof style indents
% ============================================================================= 
% No indent in proofs
\makeatletter
    \renewenvironment{proof}[1][\proofname]{\par
      \pushQED{\qed}%
      \normalfont \topsep6\p@\@plus6\p@\relax
      \trivlist
      \itemindent\z@ % original has \normalparindent
      \item[\hskip\labelsep
            \scshape
        #1\@addpunct{.}]\ignorespaces
    }{%
      \popQED\endtrivlist\@endpefalse
    }
\makeatother

% =============================================================================
% Leader dots in toc
% =============================================================================
% Creates dots leading from toc entries to relevant page number
%\makeatletter
%\renewcommand\@pnumwidth{1em} % <-- depending on the total number of pages
%\patchcmd{\@tocline}
%  {\hfil}
%  {\leaders\hbox{\,.\,}\hfil}
%  {}{}
%\makeatother

% =============================================================================
% Center sections
% =============================================================================
% somehow, once, sections appeared uncentered, and this fixes that
%\makeatletter
%    \def\section{\@startsection{section}{1}%
%      \z@{.7\linespacing\@plus\linespacing}{.5\linespacing}%
%      {\normalfont\bfseries\centering}}
%\makeatother

% =============================================================================
% Center subsections
% =============================================================================
% creates a separate line for subsection titles and centers them
%\makeatletter
%    \def\subsection{\@startsection{subsection}{2}%
%      \z@{.5\linespacing\@plus.7\linespacing}{.25\linespacing}%
%      {\normalfont\bfseries\centering}}
%\makeatother