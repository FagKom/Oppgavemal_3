\chapter{Chapter title}

In this chapter, I'll give some examples on how I use this template. Feel free to look at \textquote{preamble.tex} inside the \textquote{setup} folder and change it as you like :) Some adjustments, like leader dots in Table of Contents, are commented out.

% ============================================================

\section{Section title}
\label{sec:sec_title}

% ============================================================

Unless otherwise stated, \(\R^n\) will always be assumed equipped with the standard topology, i.e. \(U \in \T_{\R^n}\) is open if \(U = \bigcup B_r(x)\). See \cite{Lee2013} for an introduction to the following chapter. 
% the shorthand \R, etc., can be found and changed in settings.tex in the setup folder

\begin{mydef}
    \label{def:manifold} 
   An \emph{$n$-dimensional topological manifold} (or \emph{$n$-manifold}) is a topological space $(M, \T)$ with the following properties:
    \begin{enumerate}
         \item \((M, \T)\) is a Hausdorff space,
         \item There exist a second-countable basis for \((M, \T)\), and
         \item M is locally Euclidean of dimension n, i.e. for any point $p\in M$ there exists a neighborhood $U\in\T$ of $p$, an open set $V\subseteq\R^n$ and a homeomorphism $\phi:U \to  V=\phi(U) \;$.
    \end{enumerate}
\end{mydef}

Here is some text between the definition and the example.

\begin{ex}
    \label{ex:manifold}
    Let $M$ and $N$ be manifolds. The set $M\times N$ can be made into a manifold by giving it the product topology.
\end{ex}

\begin{remark}
    Remarks are remarkable.
\end{remark}

From \cref{ex:manifold}, we can construct new manifolds from old ones.
\lipsum[1]

\begin{lemma}
    \label{lem:manifold}
    Connected manifolds are path-connected.
\end{lemma}
\begin{proof}
    See \cite{Lee2013} for a proof of this.
\end{proof} 

We can use \cref{lem:manifold} to prove the next theorem.

\begin{thm}
    \label{thm:manifolds}
    Manifolds are manifolds.
\end{thm} 
\begin{proof}
    Use the definition of manifold (see \cref{def:manifold}) and \cref{lem:manifold}
\end{proof}

Because of \cref{thm:manifolds}, we have the following.

\begin{cor}
    Manifolds are more than topological spaces.
\end{cor}
\begin{proof}
    Follows directly.
\end{proof}

\subsection{Subsection title}
Yoneda lemma is useful for dealing with generalized manifolds. 
\lipsum[1]
 
\begin{thm}[Yoneda lemma]
    For any presheaf $\F$, evaluation on $X$ determines a bijection $\Pre(\F_X , \F) \cong \F(X)$ of sets.
\end{thm}  